\documentclass[dvipsnames]{beamer}
% \documentclass[handout,dvipsnames]{beamer} 

\mode<presentation> {

%\usetheme{default}
%\usetheme{AnnArbor}
%\usetheme{Antibes}
%\usetheme{Bergen}
%\usetheme{Berkeley}
%\usetheme{Berlin}
%\usetheme{Boadilla}
%\usetheme{CambridgeUS}
%\usetheme{Copenhagen}
%\usetheme{Darmstadt}
%\usetheme{Dresden}
%\usetheme{Frankfurt}
%\usetheme{Goettingen}
%\usetheme{Hannover}
%\usetheme{Ilmenau}
%\usetheme{JuanLesPins}
%\usetheme{Luebeck}
\usetheme{Madrid}
%\usetheme{Malmoe}
%\usetheme{Marburg}
%\usetheme{Montpellier}
%\usetheme{PaloAlto}
%\usetheme{Pittsburgh}
%\usetheme{Rochester}
%\usetheme{Singapore}
%\usetheme{Szeged}
%\usetheme{Warsaw}

%\usecolortheme{albatross}
%\usecolortheme{beaver}
%\usecolortheme{beetle}
%\usecolortheme{crane}
\usecolortheme{dolphin}
%\usecolortheme{dove}
%\usecolortheme{fly}
%\usecolortheme{lily}
%\usecolortheme{orchid}
%\usecolortheme{rose}
%\usecolortheme{seagull}
%\usecolortheme{seahorse}
%\usecolortheme{whale}
%\usecolortheme{wolverine}

\setbeamertemplate{navigation symbols}{} 
}

%\usepackage{etex}
\usepackage[]{xcolor}
\usepackage{amsfonts}
\usepackage{graphicx} 
%\usepackage{booktabs} 
\usepackage{tikz}
\usetikzlibrary{shapes.symbols,matrix,arrows,decorations.pathmorphing,calc,shapes}
\usetikzlibrary{calc}
\usepackage[matrix,arrow]{xy}
 
 \usepackage{braids}
% \usepackage{wrapfig}
 \usepackage{adjustbox}
 %\usepackage[colorlinks,plainpages,backref,urlcolor=blue]{hyperref}
\usepackage{bbm}
\usepackage{mathrsfs}
\usepackage{xstring}
\usepackage{array}
 \usepackage{mathtools}
 
% \newtagform{blue}{\color{blue}(}{)}
 
 % \usepackage{ragged2e}
%\let\olditem=\item% 
%\renewcommand{\item}{\olditem \justifying}%

\newcounter{sarrow}
\newcommand\xlrsquigarrow[1]{%
\stepcounter{sarrow}%
\mathrel{\begin{tikzpicture}[baseline= {( $ (current bounding box.south) + (0,-0.5ex) $ )}]
\node[inner sep=.5ex] (\thesarrow) {$\scriptstyle #1$};
\path[draw,<->,>=angle 90,decorate,
  decoration={zigzag,amplitude=0.7pt,segment length=1.2mm, pre=lineto, pre length=3pt,post length=3pt }] 
    (\thesarrow.south west)-- (\thesarrow.south east);
\end{tikzpicture}}%
} 


 \newcommand\xrsquigarrow[1]{%
\stepcounter{sarrow}%
\mathrel{\begin{tikzpicture}[baseline= {( $ (current bounding box.south) + (0,-0.5ex) $ )}]
\node[inner sep=.5ex] (\thesarrow) {$\scriptstyle #1$};
\path[draw,->,>=angle 90,decorate,
  decoration={zigzag,amplitude=0.7pt,segment length=1.2mm, pre=lineto, pre length=3pt,post length=3pt }] 
    (\thesarrow.south west)-- (\thesarrow.south east);
\end{tikzpicture}}%
} 
 
  
\newtheorem{prop}[theorem]{Proposition}

\theoremstyle{example}
\newtheorem{remark}[theorem]{Remark}
\newtheorem{conjecture}[theorem]{Conjecture}
\newtheorem{blankthm}[theorem]{ }

\DeclareMathOperator{\im}{im}
\DeclareMathOperator{\id}{id}
\DeclareMathOperator{\codim}{codim}
\DeclareMathOperator{\coker}{coker}
\renewcommand{\hom}{\operatorname{Hom}}
\DeclareMathOperator{\Supp}{Supp}
\DeclareMathOperator{\Ann}{Ann}
\DeclareMathOperator{\rank}{rank}
\DeclareMathOperator{\Ext}{Ext}
\DeclareMathOperator{\Tor}{Tor}
\DeclareMathOperator{\ab}{ab}
\DeclareMathOperator{\abf}{abf}
\DeclareMathOperator{\lin}{lin}
\DeclareMathOperator{\Lie}{Lie}
\DeclareMathOperator{\gr}{gr}
\DeclareMathOperator{\Sym}{Sym}
\DeclareMathOperator{\Hilb}{Hilb} 
\DeclareMathOperator{\TC}{TC}
\DeclareMathOperator{\ad}{ad}
\DeclareMathOperator{\torsion}{torsion}
\DeclareMathOperator{\GL}{GL}
\DeclareMathOperator{\Spec}{Spec}
\DeclareMathOperator{\Aut}{Aut}
\DeclareMathOperator{\IA}{IA}
\DeclareMathOperator{\Conf}{Conf}
\DeclareMathOperator{\Mod}{Mod}

\newcommand{\dga}{\ensuremath{\textsc{dga}}}
\newcommand{\cdga}{\ensuremath{\textsc{cdga}}}
\newcommand{\dgla}{\ensuremath{\textsc{dgla}}}
 
\newcommand{\R}{\mathbb{R}}
\newcommand{\Q}{\mathbb{Q}}
\newcommand{\C}{\mathbb{C}}
\newcommand{\Z}{\mathbb{Z}}
\newcommand{\F}{\mathbb{F}}
\newcommand{\T}{\mathbb{T}}
\newcommand{\kk}{\mathbb{Q}}
\newcommand{\mm}{\mathbbm{m}}

\newcommand{\fA}{\mathfrak{A}}
\newcommand{\fb}{\mathfrak{b}}
\newcommand{\fB}{\mathfrak{B}}
\newcommand{\fh}{\mathfrak{h}}
\newcommand{\fg}{\mathfrak{g}}
\newcommand{\fn}{\mathfrak{n}}
\newcommand{\fM}{\mathfrak{M}}
\newcommand{\fm}{\mathfrak{m}}
\newcommand{\fL}{\mathfrak{L}}
\newcommand{\tr}{\mathfrak{tr}}
\newcommand{\fp}{\mathfrak{p}}

\newcommand{\bL}{\mathbf{L}}
\newcommand{\bx}{\mathbf{x}}
\newcommand{\br}{\mathbf{r}}
\newcommand{\by}{\mathbf{y}}
\newcommand{\bw}{\mathbf{w}}

\renewcommand{\k}{\mathbb{Q}}

\newcommand{\cD}{\mathcal{D}}
\newcommand{\cP}{\mathcal{P}}
\newcommand{\cB}{\mathcal{B}}
\newcommand{\cW}{\mathcal{W}}
\newcommand{\cR}{\mathcal{R}}
\newcommand{\cV}{\mathcal{V}}
\newcommand{\cM}{\mathcal{M}}
\newcommand{\cC}{\mathcal{C}}
\newcommand{\cF}{\mathcal{F}}

\newcommand{\vB}{vB}
\newcommand{\vP}{vP}
\newcommand{\wB}{wB}
\newcommand{\wP}{wP}
\newcommand{\PV}{PV}
\newcommand{\SB}{SB}
\newcommand{\SG}{SG}
\newcommand{\UB}{UB}
\newcommand{\TB}{TB}
%\newcommand{\wP}{P\Sigma}
\newcommand{\s}{s}

\newcommand{\surj}{\twoheadrightarrow}
\newcommand{\inj}{\hookrightarrow}
\newcommand{\abs}[1]{\left| #1 \right|}

\newcommand{\brak}[1]{\ensuremath{\left\{ #1 \right\}}}
\newcommand{\set}[2]{\ensuremath{\Set{ #1 \, | \, #2}}}
\newcommand{\setang}[2]{\ensuremath{\Braket{#1 \, | \, #2}}}
 
\newcolumntype{M}[1]{>{\centering\arraybackslash}m{#1}}
\newcolumntype{L}[1]{>{\arraybackslash}m{#1}}
\newcolumntype{N}{@{}m{0pt}@{}}

\definecolor{gold}{rgb}{0.83, 0.69, 0.22}

%-------------------------------------------------------------------- 
%	TITLE PAGE
%------------------------------------------------------------------- 

\title[\scalebox{.88}{Matrix methods in algebraic topology}]{Matrix methods in algebraic topology} 
% The short title appears at the bottom of every slide, the full title is only on the title page

\author[He Wang]{ \large He Wang   \and \\  \small (joint work with Alex Suciu)}

% - Give the names in the same order as the appear in the paper.
% - Use the \inst{?} command only if the authors have different
%   affiliation.

\institute[UNR ] % (optional, but mostly needed)
{\Large
  \inst{University\, of\, Nevada,\, Reno }
  \and
  \inst{ }
  {  \\  \small
The 8th International Conference on Matrix Analysis and Applications\\ 
 (ICMAA 2019) \\
\small University of Nevada, Reno
   }}


\date{ \scalebox{.88} {July 17, 2019}}
 

\begin{document}

%--------------------------------------------------- 
%	Frame1
%---------------------------------------------------- 

\begin{frame} 
\titlepage 
\end{frame}


%------------------------------------------------
%Frame2
%------------------------------------------------

%\begin{frame}
%\frametitle{Overview}  
%\tableofcontents  
%\end{frame}

%--------------------------------------------------- 
%	PRESENTATION SLIDES
%--------------------------------------------------

%------------------------------------------------
%\section{Background}
%------------------------------------------------

%------------------------------------------------
\section{Formality Properties}  
%------------------------------------------------

\begin{frame}\small
\frametitle{Topology}

\begin{itemize}

\item \textbf{\textcolor{Black}{Topological spaces:}} Geometric objects like lines, planes, Euclidean Spaces $\R^n$,
spheres, torus, Mobi\"{u}s band, Klein bottle, manifolds, metric spaces, etc.


\pause


\item \textbf{\textcolor{Red}{Goal:}}  Classification of all topological spaces up to homeomoprhism.\\
 


\pause

\item \textbf{Example:}  Let $M$ be a smooth compact $n$-manifold homotopy-equivalent to $S^n$. 
Then $M$ is homeomorphic to $S^n$.   \pause  (Generalized Poinc\'{a}re conjecture)
 

Three \textcolor{gold}{Fields Medals} were awarded:
Smale (1966), Freedman (1986), and Perelman (2006) for solving the problem
for $n\geq 5$, $n=4$ and $n=3$. 
 
\pause

\item \textbf{\textcolor{Blue}{An ``easier'' goal:}}  Classification of all topological spaces up to  homotopy equivalence.  

\pause

\item \textbf{\textcolor{Green}{An even ``easier'' goal:}}  Classification of all topological spaces up to rational homotopy equivalence. 

\end{itemize}
\end{frame}

%------------------------------------------------
% Frame  
%------------------------------------------------

\begin{frame}  \small
\frametitle{Algebraic Topology}
 
A  \textcolor{Green}{basic goal} in algebraic topology: \\  
{\qquad \em ``Find \textcolor{Red}{algebraic invariants} that classify topological spaces\\
\qquad \ \  up to homeomorphism, or up to (rational) homotopy equivalence.''}\\ \

\pause

Let $X$ be a connected CW-complex (topological space). 

 

\begin{itemize}
\item  \textcolor{Blue}{Fundamental group $\pi_1(X)$.}
\item  \textcolor{Blue}{Homology groups $H_*(X;\Q)$  $\Longrightarrow$
  Betti numbers $b_i(X)$.}
\item  \textcolor{Blue}{Cohomology algebra $H^*(X;\Q)$. }
\pause
\item \textcolor{Green}{Higher (rational) homotopy groups $\pi_*(X)\otimes \Q$.}
\item \textcolor{Green}{Homotopy Lie algebra $\pi_*(\Omega X)\otimes \Q$.}
%\item \textcolor{Red}{Higher Massey products.}
%\item \textcolor{Red}{Cohomology operations.}
\item ...... 
\pause
\item  \textcolor{Red}{Lie algebras/ varieties in this talk.}
\item  \textcolor{Red}{Higher products on cohomology in a current project with Chris Rogers. }
\end{itemize}

\end{frame}


%------------------------------------------------
% Frame  
%------------------------------------------------

\begin{frame} 
\frametitle{Relation with group theory}

\textbf{\textcolor{Green}{Groups:}}  $\Z$, $\Z^n$, $\Z/p\Z$, $S_n$, $F_n$, ...
\vspace{0.1cm}


\textbf{\textcolor{Red}{Goal:}}  Classification of all groups up to isomorphism.

\pause

\begin{itemize}
\item $X$: a connected CW-complex with finitely many $1$-cells.

\item $\textcolor{Green}{G:=\pi_1(X)}$, \textbf{\textcolor{Green}{a finitely generated group}}.
\pause
\item $K(G,1)$: the Eilenberg-MacLane space.
 
$$\textcolor{Green}{G}  \xlrsquigarrow{\quad 1-1\quad }  K(G,1)\qquad\qquad \ \ $$

 
\pause 

\item The \textit{\textcolor{Red}{cohomology algebra}}\/ of $G$, $H^*(G;\Q):=H^*(K(G,1);\Q)$.
\end{itemize}  
\end{frame}






\begin{frame}
\frametitle{Am matrix group example}
\begin{example}[Heisenberg group]

\begin{itemize}
\item The Heisenberg manifold,  
$M=\left\{ 
\begin{bmatrix}
1 & a & c\\
0 & 1& b\\
0&0&1
\end{bmatrix} ~\middle| ~
a,b,c\in \R
\right\}.$\\

\vspace{0.2cm}

\item Then $\mathcal{H}=\pi_1(M)=\left\{ 
\begin{bmatrix}
1 & a & c\\
0 & 1& b\\
0&0&1
\end{bmatrix} ~\middle| ~
a,b,c\in \Z
\right\}$, and $M=K(\mathcal{H},1)$.\\
\vspace{0.2cm}

\pause

\item 
$\mathcal{H}=\langle x,y,z \mid xyx^{-1}y^{-1}z^{-1},~ xzx^{-1}z^{-1},~ yzy^{-1}z^{-1}\rangle$.\\
%is a 3-manifold satisfying an orientable fiber bundle $S^1\to M\to S^1\times S^1$.
\vspace{0.2cm}

\pause
 
\item Rational cohomology: $H^1=\Q^2$ with basis $\{u_1, u_2\}$; $H^2=\Q^2$ with basis $\{v_1, v_2\}$; $H^3=\Q$ with basis $\{w\}$.\\
\vspace{0.2cm}
 
 
\item Nontrivial cup products: $u_1\cup v_1=u_2\cup v_2=w$.
\end{itemize}
\end{example}
\end{frame}
  
  
  
   
 
 
 
%------------------------------------------------
% Frame 30
%------------------------------------------------
\begin{frame}
\frametitle{References}
\footnotesize{
\begin{thebibliography}{99}  

 	
\bibitem{SW-holo}
Alexander~I. Suciu and He~Wang,  
{\em Cup products, lower central series, and holonomy Lie algebras}, 
J. Pure. Appl. Algebra \textbf{223} (2019), no.~8, 3359--3385.

\bibitem{SW-formal}
Alexander~I. Suciu and He~Wang, 
\emph{Formality properties of finitely generated 
groups and Lie algebras}, Forum Math. \textbf{31} (2019), no. 4, 867--905

\bibitem{SW-mccool}
Alexander~I. Suciu and He~Wang, \emph{Chen ranks and resonance varieties 
of the upper McCool groups}, to appear in  Advances in Applied Mathematics. 
 
 
 

 
 
 
\end{thebibliography}
}
\vspace{0.5cm}
 
\Huge{\centerline{\it \textcolor{red}{Thank You!}}}
\end{frame}

 
  
%----------------------------------------------------------------------------------------
 
\end{document} 
 
 
 
  
 